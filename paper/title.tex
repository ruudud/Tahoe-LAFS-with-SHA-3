% Latex-versjon av ITEM rapportmal.
% Lagd av <lasse.karstensen@gmail.com>, desember 2009.
% Lisens: public domain. 
%
\begin{titlepage}
\begin{center}
\textsc{NORWEGIAN UNIVERSITY OF SCIENCE AND TECHNOLOGY\\
FACULTY OF  INFORMATION TECHNOLOGY, MATHEMATICS AND ELECTRICAL ENGINEERING} \\
\vspace{0.5cm} 
% crop-et fra http://www.ntnu.no/infoavdelingen/selvhjelp/logoer/ntnu/NTNU_engelsk_RGB.png
\includegraphics[scale=0.5]{NTNU-logo.png} \\

\vspace{1.0cm}
{\Huge{PROJECT ASSIGNMENT}}
\vspace{1.0cm}

\begin{tabular}{ p{4cm} p{11cm}}

Students' name:	& Eirik Haver and Pål Ruud \\
Course: & TTM13 \\
Title: & Experimenting with SHA-3 candidates in Tahoe-LAFS \\
%\vspace{1cm}
Description: & \\
\end{tabular}
{\small{\begin{tabular}{p{15cm}}
\vspace{0.2cm}

Tahoe-LAFS is a Free Software/Open Source decentralized data store. It
distributes your file system across multiple servers, and even if some of the
servers fail or are taken over by an attacker, the entire file system continues
to work correctly and to preserve your privacy and security.
\\\\
One of the basic security components used in Tahoe-LAFS is the cryptographic
hash function SHA-256.
\\\\
In the light of the worldwide SHA-3 hash competition, this task is about
making a reproducible, automated benchmark which shows how the performance of
Tahoe-LAFS is affected by the performance of the different SHA-3 candidate hash
functions. Before any testing can be done, Python bindings to the C
implementations of the SHA-3 candidates have to be made, since Tahoe-LAFS is
written in the Python programming language.
\\\\
\end{tabular}  }}

\begin{tabular}{ p{4cm} p{11cm}}
Deadline: & 2010-12-20 \\
Submission date: & 2010-12-xx \\
Department: & Department of Telematics \\
Supervisor: & Danilo Gligoroski \\\\
\end{tabular}
\vspace{0.5cm}

Trondheim, \today 

\vspace{0.4cm}
\line(1,0){150} \\
Danilo Gligoroski, NTNU/ITEM. 

\end{center}
\end{titlepage}

