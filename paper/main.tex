\documentclass[english,12pt,a4paper]{book}
\usepackage[T1]{fontenc} % In case we want special characters
\usepackage[utf8]{inputenc} % We are all writing in UTF-8

\usepackage{appendix} % Fixes formatting of appendices
\usepackage{graphicx} % We *may* use images
\usepackage{hyperref} % Internal and external links is nice
\hypersetup{pdfborder=0 0 0} % ..especially without red borders

% We need this to make it easier to ensure we start chapters on odd pages.
\newcommand{\ensureodd}{
    \newpage
%    \thispagestyle{empty}
    \mbox{}
}

\author{Eirik Haver \and Pål Ruud}
\title{Project assignment - Tahoe-LAFS with SHA-3 candidates}
\date{\today}

\begin{document}

% Latex-versjon av ITEM rapportmal.
% Lagd av <lasse.karstensen@gmail.com>, desember 2009.
% Lisens: public domain. 
%
\begin{titlepage}
\begin{center}
\textsc{NORWEGIAN UNIVERSITY OF SCIENCE AND TECHNOLOGY\\
FACULTY OF  INFORMATION TECHNOLOGY, MATHEMATICS AND ELECTRICAL ENGINEERING} \\
\vspace{0.5cm} 
% crop-et fra http://www.ntnu.no/infoavdelingen/selvhjelp/logoer/ntnu/NTNU_engelsk_RGB.png
\includegraphics[scale=0.5]{NTNU-logo} \\

\vspace{1.0cm}
{\Huge{PROJECT ASSIGNMENT}}
\vspace{1.0cm}

\begin{tabular}{ p{4cm} p{11cm}}

Student's name:	& Eirik Haver and Pål Ruud \\
Course: & TTM13 \\
Title: & Experimenting with SHA-3 candidates in Tahoe-LAFS \\
%\vspace{1cm}
Description: & \\
\end{tabular}
{\small{\begin{tabular}{p{15cm}}
\vspace{0.2cm}
Lorem ipsum dolor sit amet, consectetur adipiscing elit. Nunc nibh quam, posuere quis dignissim vel, pulvinar vel leo. Morbi pellentesque est a magna rutrum sollicitudin. Duis nec orci porttitor tellus viverra aliquet gravida nec tellus. Aenean aliquet nulla sit amet metus sodales mattis. In pulvinar euismod consectetur. Maecenas felis nunc, suscipit ut semper et, consequat et est. 
\\\\
Donec vitae ipsum tortor. Nulla at tincidunt erat. Nunc vel risus vel mi ornare molestie non at nunc. Phasellus posuere cursus odio, nec consectetur dolor tempus quis. Proin non erat erat. Aenean et leo eros, et congue sem. Nunc auctor porta magna, ac fermentum augue malesuada vel. Nam aliquet augue eu nibh facilisis ac placerat felis vehicula. Quisque faucibus placerat vulputate. Mauris nunc turpis, venenatis ac tincidunt condimentum, tincidunt id magna. Mauris aliquet convallis accumsan. 
\\\\
\end{tabular}  }}

\begin{tabular}{ p{4cm} p{11cm}}
Deadline: & 2010-12-xx \\
Submission date: & 2010-12-xx \\
Department: & Department of Telematics \\
Supervisor: & Danilo Gligoroski \\\\
\end{tabular}
\vspace{0.5cm}

Trondheim, \today 

\vspace{0.4cm}
\line(1,0){150} \\
Danilo Gligoroski, NTNU/ITEM. 

\end{center}
\end{titlepage}


\pagestyle{empty}

\chapter*{Abstract}
\addcontentsline{toc}{chapter}{Abstract}
\pagestyle{plain}
\pagenumbering{Roman}
\setcounter{page}{1}

\chapter*{Preface}
\addcontentsline{toc}{chapter}{Preface}

\tableofcontents

\addcontentsline{toc}{chapter}{\listfigurename} % Manual, because report style
\listoffigures

\addcontentsline{toc}{chapter}{\listtablename}
\listoftables

\chapter*{Acronyms}
\addcontentsline{toc}{chapter}{Acronyms}

% Insert acronyms here

\chapter{Introduction}
\pagenumbering{arabic}
\setcounter{page}{1}

Thorough problem description

\section{Method}

Technical procedure, Cython, two persons, test environment, how we test,
GitHub?

\section{Scope and objectives}

Should we include this?

\section{Outline}

What follows in this document, chapter by chapter


\chapter{Background technologies}
% Find new and better title

\section{Cryptographic Hash Functions}
A Cryptographic Hash function is a deterministic mathematical procedure which
takes an arbritary block of data and outputs a fixed-size bit string. The
output is refered to as the hash value, message digest or simply digest.
Another property of a cryptographic hash function is that a small change in the
input data (just 1 bit) should completly change the output of the hash
function. In other words it should be infeasible to find the reverse of a
cryptographic hash function. It should also be infeasible to find two blocks of
data which produce the same hash value (a collision).

\subsection{NIST SHA-3 Competition}
% Hash functions, NIST Comptetition, Candidates, SuperCop?
The Secure Hash Function version 3 (SHA-3) is a comming standard cryptographic
hash function set to replace the current standards SHA-1 and SHA-2.
It will be selected from a competition held by the National
Institute of Standards and Technology (NIST). The competition started in 2006
with submission deadline for candidates at the end of 2008. 51 candidates made
it into round 1, however just 14 passed to the second round. The reason for not
passing round 1 are either design weaknesses, substantial cryptographic
weaknesses or performance issues. The winner is sceduled to be selected by the 
2nd quarter of 2012.

The submissions for the various rounds of the SHA-3 competition include
reference implementations, optimized implementations, test-vectors to verify
implementations and documentations explaining the submission.

\subsection{SuperCop}
Supercop is a toolkit developed by VAMPIRE lab for measuring performance of
cryptographic software, among them the SHA-3 candidates. While security is a
primary goal of a hash function, performance can not be neglected if the hash
function should be taken into use. The difference between the implementations
in the official submissions and in supercop is first that the supercop
implementation are continiously updated and that there are more specific
hardwareimplementations available in supercop.

Results by the supercop are published at (URL).


\section{Tahoe-LAFS}
%What is it, how/where are hash functions used, Python, pycryptopp
%comparison with RAID-6

The Tahoe Least-Authority Filesystem is a system for secure,
distributed data storage \cite{t_tahoe}. Tahoe was originally developed with
funding from the former web backup service provider Allmydata, but is now a
stand-alone Open Source project \cite{t_ars}.

New files saved using Tahoe follow this pattern:
\begin{enumerate}
\item The file is encrypted using AES.
\item The file is split up into \emph{N} parts using \emph{erasure coding}.
\item The parts are distributed to \emph{N} storage servers.
\end{enumerate}

The use of erasure coding enables Tahoe to recover a file using only a
predefined subset of the parts distributed to the storage servers.

% TODO: comparison with RAID?

\subsection{Advantages using a secure, distributed file system}
% Use cases?


\section{Python bindings using Cython}
%What is it, how, why, alternatives?
% TODO: Hvorfor er nedenfornevnte i ASM eller C?
TAHOE-LAFS is written in python, but the SHA-3 candidate implementations found
in both the official NIST submissions and supercop are either in C or in
assembly. Therefore we need a way to make the implementations available in
Python. There a several ways to do this. Python itself includes a C-API,  
other options include SWIG, Cython and possible others.

\subsection(Python C-API)
The Python C-API is the official method of creating C-bindings for Python. The
procedure to do this includes writing entrypoints or wrappers for your C
functions in C. Then, through compiling you can acess your C-libraries and
functions through Python. The drawback with this approach, when faced with an
already existing C-library is that you have to actually write the
wrapper-functions or modify the C-library.

\subsection{Cython}
Cython tries to solve the problem of binding C-libraries and functions by the
use of Python instead of the use of C which the official API does. This is done
by only defining the entrypoints to the required C-functions and datatypes in a
Python/C/C++ hybrid syntax. Cython also allows creating new C/C++ functions,
available or not available to Python, in the same syntax. Cython works by
transforming the hybrid syntax into C/C++-code which utilises the official
Python C-API and compiling that. An expected tradeoff by using Cython is a
performance drop due to the extra layer of abstraction.


\chapter{Procedure}

TODO: We have to figure out a new title
Keywords: Python bindings in detail, unit tests, using it in Tahoe-LAFS

A bit more technical description of how we tested the candidates in Tahoe-LAFS

Limitations/assumptions

Optimized candidates.


\chapter{Results}

What is tested and what is not tested?

tables, graphs?


\chapter{Discussion}

What could have been done differently?

Compare results with general (supercop) results.

Optimized candidates.

\cite{s_nistround2}


\chapter{Conclusion}

A somewhat short conclusion summing up results and discussion

\section{Future work}

New SHA-3 capability in Tahoe-LAFS (GSOC ref) to get our code to trunk


% BibTeX bibliography lives in external file
\bibliographystyle{plain}
\bibliography{ref_sha3,ref_tahoe}

\appendix
\appendixpage
\addappheadtotoc

% Use ordinary \chapters from here on..

\end{document}
