\documentclass[screen]{beamer}
\usepackage[T1]{fontenc}
\usepackage[utf8]{inputenc}

%
%   For at dette skal funke:
%   wget http://www.ntnu.no/selvhjelpspakken/ntnubeamer/beamerntnu.tar.gz
%   extract til: /usr/share/texmf/tex/latex/beamer/themes
%   kjør sudo mktexlsr
%   sourcE: http://www.ntnu.no/selvhjelpspakken/ntnubeamer/

% Bruk NTNU-temaet for beamer (her i bokmålvariant), alternativer er
% ntnunynorsk og ntnuenglish.
\usetheme{ntnuenglish}
 
% Angi tittelen, vi gir også en kortere variant som brukes nederst på
% hver slide:
\title[SHA-3 in Tahoe-LAFS]%
{Experimenting with SHA-3 candidates in Tahoe Least Authorithy File System}

% Denne kan du også bruke hvis det passer seg:
%\subtitle{Valgfri undertittel}

% Angir foredragsholder, også en (valgfri) kortversjon i
% hakeparanteser først som kommer nederst på hver slide:
\author[EH \& PR]{Eirik Haver and Pål Ruud}

% Institusjon. Bruk gjerne disse slik det passer best med det du vil
% ha.  Valgfri kortversjon her også
\institute[NTNU]{Department of Telematics}

% Datoen blir også trykket på forsida. 
\date{X. December 2010}
%\date{} % Bruk denne hvis du ikke vil ha noe dato på forsida.


% Fra her av begynner selve dokumentet
\begin{document}

% Siden NTNU-malen har en annen bakgrunn på forsida, må dette gjøres
% i en egen kommando, ikke på vanlig beamer-måte:
\ntnutitlepage

% Her begynner første slide/frame, (nummer to etter forsida). 
\begin{frame}
  \frametitle{Tahoe-LAFS}
  \begin{itemize}
  \item Distributed FS
  \item Principle of Least Authority
  \item Erasure coding, k-out-of-n
  \item Written in Python
  \item Uses (double) SHA256-today
    \begin{itemize}
        \item Because of Length-Extension-Attack
    \end{itemize}
  \end{itemize}
\end{frame}

\begin{frame}
  \frametitle{NIST SHA-3 Competition}
  \begin{itemize}
  \item Select SHA-3 implementation by 2nd Quarter 2012
  \item Currently 14 candidates, one from NTNU
  \item Resistant to Length-Extension-Attacks
  \end{itemize}
\end{frame}

\begin{frame}
  \frametitle{What we have done}
  \begin{enumerate}
  \item Made Python-binding for all candidates (C implementations) 
  \item Modified Tahoe-LAFS to use the bindings
  \item Benchmarked Tahoe-LAFS with the different candidate implementations
  \end{enumerate}
\end{frame}

\begin{frame}
  \frametitle{Python-bindings}
  \begin{itemize}
  \item Optimized version of candidates available at eBACS: ECRYPT Benchmarking
  of Cryptographic Systems.
  \item We have used the fastest, capable C-version available for our target
  architecture, 32-bit.
  \item Made them available in Python by the use of Cython in ``SHA3Lib''
  \end{itemize}
\end{frame}

\end{document}
